% LaTeX resume using res.cls
\documentclass[line,margin]{res}
%\usepackage{helvetica} % uses helvetica postscript font (download helvetica.sty)
%\usepackage{newcent}   % uses new century schoolbook postscript font
\usepackage{hyperref}
\usepackage{xcolor}
\usepackage[utf8]{inputenc}
\usepackage{enumitem}
\usepackage{fancyhdr}
\setlength{\headheight}{13.6pt}
\pagestyle{fancy}

\fancypagestyle{plain}{ %
  \fancyhf{} % remove everything
  \renewcommand{\headrulewidth}{0pt} % remove lines as well
  \renewcommand{\footrulewidth}{0pt}
}

\hypersetup{%
  colorlinks=true,% hyperlinks will be coloured
  urlcolor=blue,% hyperlink text will be green
  urlbordercolor=red,% hyperlink border will be red
}

\begin{document}
\thispagestyle{empty}

\name{{\Large \normalfont Résumé for }Erik Wilson}
\email{Erik.E.Wilson@gmail.com}
\lfoot{\href{mailto:Erik.E.Wilson@gmail.com}{Erik.E.Wilson@gmail.com}}

\renewcommand{\headrulewidth}{0pt}

\begin{resume}

  \section{CAREER\\SUMMARY}
  Software Engineer with many years of experience working with teams of varying size, knowledge, and locality.
  Exposure to a wide variety of languages, architectures, operating systems, libraries, and supporting
  software, from national supercomputer systems to microprocessors. Specialize in researching and developing
  solutions for difficult problems in business and scientific software design. Always expanding breadth and
  depth of knowledge while seeking excellence in engineering.

  \section{SKILL SET\\SUMMARY}
          {\bf \emph{Languages}:} Go, Python, Rust, C, C++, Lua, Javascript, Java, Perl, Shell \\
          {\bf \emph{Software}:} Kubernetes, Docker and ContainerD, Linux, MacOS, Windows, CI/CD

  \section{WORK\\EXPERIENCE}
          {\bf \emph{Software Engineer}} \hfill Tempe, AZ

          {\sl HPE - Determined AI} \hfill 2022 - 2024
          \begin{itemize} \itemsep -2pt
          \item
            Backend development on Determined AI training platform using Go, as standalone service or utilizing Kubernetes or Slurm deployments.
          \item
            Go and Python development for Determined connected services performing Retrieval Augmented Generation and Large Language Model inference in Kubernetes clusters.
          \end{itemize}

          {\sl Red Canary} \hfill 2022
          \begin{itemize} \itemsep -2pt
          \item
            Main platform development for sensor analysis using Ruby, multi-arch migration of Kubernetes services to reduce operational costs.
          \item
            Linux EDR sensor development to inspect processes and network data in Rust utilizing eBPF, and development of connected Go telemetry services.
          \end{itemize}

          {\sl SUSE - Rancher} \hfill 2018 - 2021
          \begin{itemize} \itemsep -2pt
          \item
            Feature development and bug fix support for Rancher server, a platform for launching and management of custom Kubernetes distributions on cloud providers.
          \item
            Developer and maintainer for K3S, an open source and lightweight Kubernetes distribution with a focus on ease of use and edge capabilities.
          \end{itemize}

          {\sl Nextiva} \hfill 2018
          \begin{itemize} \itemsep -2pt
          \item
            Frontend development with React and Javascript, implementing design requirements for user identity and 
            access management administration.
          \item
            Backend support with Python and Java, code reviews, debugging, and bug fixes.
          \end{itemize}

          {\sl Citrix - Octoblu Inc} \hfill 2014 - 2017
          \begin{itemize} \itemsep -2pt
          \item
            Assisted in the architecture and development of an Internet of Things platform.
          \item
            Frontend development with Angular and React, backend development using Docker with Javascript (Node.js, CoffeeScript, ES6).
            Device integration with a variety of languages, including: Node.js, Lua, Java, Groovy, and C++.
          \end{itemize}

          {\sl Iced Development} \hfill 2012 - 2014
          \begin{itemize} \itemsep -2pt
          \item
            Software engineering services for a global Advance Deposit Wagering platform. Use of RedHat Enterprise Linux, Apache, Tomcat, MySQL, MongoDB, Java, JavaScript, and Node.
          \item
            Development of essential services for product launch with integration of partner totalizer and deposit systems. Bottleneck and optimizations analysis and resolution post-launch.
          \end{itemize}

          \newpage

          {\bf \emph{Software Engineer}, Boston University} \hfill Boston, MA

          {\sl Center for Integrated Space Weather Modeling (CISM)}, Astronomy \hfill 2005 - 2013
          \begin{itemize} \itemsep -2pt %reduce space between items
          \item
            Port various models and scientific packages to other platforms (e.g. LLNL's A++P++, Overture, and PnMPI to AIX and Cray; IBM's OpenDX to OSX).
          \item
            Maintain core infrastructure of hardware and software.
          \end{itemize}

          {\sl John Lyon (LFM)}, Astronomy \hfill 2008 - 2013
          \begin{itemize} \itemsep -2pt %reduce space between items
          \item
            Performance analysis and optimizations of the Lyon-Fedder-Mobarry (LFM) magnetohydrodynamics model.  Analysis requires performance reviews of the LFM model and associated libraries on various national supercomputer systems.
          \item
            Developed a C++ I/O library to utilize parallel file systems on supercomputer platforms, allowing unified access to HDF4 or HDF5 through a common API. Optional A++/P++ support allows for data super-domains.
          \end{itemize}

          {\sl Harlan Spence (NASA/LRO/CRaTER)}, Astronomy \hfill 2007 - 2010
          \begin{itemize} \itemsep -2pt
          \item
            Creation of TCP/UDP socketed Perl server used to decompose, calibrate, and redistribute real-time network data from the Cosmic Ray Telescope for the Effects of Radiation (CRaTER) instrument on the Lunar Reconnaissance Orbiter (LRO).
          \item
            Refactor of C++ data pipeline used on raw data received from the Mission Operations Center to create calibrated multi-level scientific data sets for scientific analysis and inclusion in NASA's Planetary Data System archive.
          \end{itemize}

          {\sl Nathan Schwadron (EMMREM)}, Astronomy \hfill 2007 - 2009
          \begin{itemize} \itemsep -2pt
          \item
            Developed a C++ I/O library for the Earth-Moon-Mars Radiation Environment Module (EMMREM), with user driver configuration and multi-processor snapshots of model simulation state, uses MPI and the NetCDF3 API.
          \end{itemize}

          {\sl David Coker Group}, Chemistry \hfill Summer 2004
          \begin{itemize} \itemsep -2pt
          \item
            Re-implementation of a FORTRAN 77 quantum monte carlo particle simulation to more compact and extensible modular Fortran 90 framework.
          \end{itemize}

  \section{EDUCATION}
          {\bf Arizona State University} \hfill Tempe, AZ\\
          {\sl Masters of Computer Science} \hfill Fall 2023 - Present\\
          {\sl Estimated graduation Fall 2025}\\
          {\sl (MCS, HLC accredited) with certificates in Cybersecurity and AI \& Machine Learning}

          {\bf Northern Arizona University} \hfill Flagstaff, AZ\\
          {\sl Baccalaureate of Science in Computer Science and Engineering} \hfill 2000 - 2004\\
          {\sl (BSCSE, ABET accredited) with minor in Linguistics}

  \section{OTHER\\EXPERIENCE}
          \emph{Electronics}
          \begin{itemize} \itemsep -2pt
          \item
            Assisted in design and development of CactusCon electronic badges, and custom Eagle CAD ULP for importing
            vector graphics from InkScape. Programming for ESP8266 and ESP32 devices using C and Lua.
          \item
            Arduino and similar microprocessor C/C++ development; analysis of sensor input and stepper motor control.
          \end{itemize}

          \emph{Volunteer Experience}
          \begin{itemize} \itemsep -2pt %reduce space between items
          \item
            Member of HeatSync Labs, a volunteer based maker community in Mesa, AZ.
          \item
            Event volunteerism, including HackPHX, NodeBots, DefCon, and CactusCon.
          \end{itemize}

          \emph{Github repository available at \href{https://github.com/erikwilson}{https://github.com/erikwilson}.}

\end{resume}
\end{document}
